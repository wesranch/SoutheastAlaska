% Options for packages loaded elsewhere
\PassOptionsToPackage{unicode}{hyperref}
\PassOptionsToPackage{hyphens}{url}
%
\documentclass[
]{article}
\usepackage{amsmath,amssymb}
\usepackage{iftex}
\ifPDFTeX
  \usepackage[T1]{fontenc}
  \usepackage[utf8]{inputenc}
  \usepackage{textcomp} % provide euro and other symbols
\else % if luatex or xetex
  \usepackage{unicode-math} % this also loads fontspec
  \defaultfontfeatures{Scale=MatchLowercase}
  \defaultfontfeatures[\rmfamily]{Ligatures=TeX,Scale=1}
\fi
\usepackage{lmodern}
\ifPDFTeX\else
  % xetex/luatex font selection
\fi
% Use upquote if available, for straight quotes in verbatim environments
\IfFileExists{upquote.sty}{\usepackage{upquote}}{}
\IfFileExists{microtype.sty}{% use microtype if available
  \usepackage[]{microtype}
  \UseMicrotypeSet[protrusion]{basicmath} % disable protrusion for tt fonts
}{}
\makeatletter
\@ifundefined{KOMAClassName}{% if non-KOMA class
  \IfFileExists{parskip.sty}{%
    \usepackage{parskip}
  }{% else
    \setlength{\parindent}{0pt}
    \setlength{\parskip}{6pt plus 2pt minus 1pt}}
}{% if KOMA class
  \KOMAoptions{parskip=half}}
\makeatother
\usepackage{xcolor}
\usepackage[margin=1in]{geometry}
\usepackage{color}
\usepackage{fancyvrb}
\newcommand{\VerbBar}{|}
\newcommand{\VERB}{\Verb[commandchars=\\\{\}]}
\DefineVerbatimEnvironment{Highlighting}{Verbatim}{commandchars=\\\{\}}
% Add ',fontsize=\small' for more characters per line
\usepackage{framed}
\definecolor{shadecolor}{RGB}{248,248,248}
\newenvironment{Shaded}{\begin{snugshade}}{\end{snugshade}}
\newcommand{\AlertTok}[1]{\textcolor[rgb]{0.94,0.16,0.16}{#1}}
\newcommand{\AnnotationTok}[1]{\textcolor[rgb]{0.56,0.35,0.01}{\textbf{\textit{#1}}}}
\newcommand{\AttributeTok}[1]{\textcolor[rgb]{0.13,0.29,0.53}{#1}}
\newcommand{\BaseNTok}[1]{\textcolor[rgb]{0.00,0.00,0.81}{#1}}
\newcommand{\BuiltInTok}[1]{#1}
\newcommand{\CharTok}[1]{\textcolor[rgb]{0.31,0.60,0.02}{#1}}
\newcommand{\CommentTok}[1]{\textcolor[rgb]{0.56,0.35,0.01}{\textit{#1}}}
\newcommand{\CommentVarTok}[1]{\textcolor[rgb]{0.56,0.35,0.01}{\textbf{\textit{#1}}}}
\newcommand{\ConstantTok}[1]{\textcolor[rgb]{0.56,0.35,0.01}{#1}}
\newcommand{\ControlFlowTok}[1]{\textcolor[rgb]{0.13,0.29,0.53}{\textbf{#1}}}
\newcommand{\DataTypeTok}[1]{\textcolor[rgb]{0.13,0.29,0.53}{#1}}
\newcommand{\DecValTok}[1]{\textcolor[rgb]{0.00,0.00,0.81}{#1}}
\newcommand{\DocumentationTok}[1]{\textcolor[rgb]{0.56,0.35,0.01}{\textbf{\textit{#1}}}}
\newcommand{\ErrorTok}[1]{\textcolor[rgb]{0.64,0.00,0.00}{\textbf{#1}}}
\newcommand{\ExtensionTok}[1]{#1}
\newcommand{\FloatTok}[1]{\textcolor[rgb]{0.00,0.00,0.81}{#1}}
\newcommand{\FunctionTok}[1]{\textcolor[rgb]{0.13,0.29,0.53}{\textbf{#1}}}
\newcommand{\ImportTok}[1]{#1}
\newcommand{\InformationTok}[1]{\textcolor[rgb]{0.56,0.35,0.01}{\textbf{\textit{#1}}}}
\newcommand{\KeywordTok}[1]{\textcolor[rgb]{0.13,0.29,0.53}{\textbf{#1}}}
\newcommand{\NormalTok}[1]{#1}
\newcommand{\OperatorTok}[1]{\textcolor[rgb]{0.81,0.36,0.00}{\textbf{#1}}}
\newcommand{\OtherTok}[1]{\textcolor[rgb]{0.56,0.35,0.01}{#1}}
\newcommand{\PreprocessorTok}[1]{\textcolor[rgb]{0.56,0.35,0.01}{\textit{#1}}}
\newcommand{\RegionMarkerTok}[1]{#1}
\newcommand{\SpecialCharTok}[1]{\textcolor[rgb]{0.81,0.36,0.00}{\textbf{#1}}}
\newcommand{\SpecialStringTok}[1]{\textcolor[rgb]{0.31,0.60,0.02}{#1}}
\newcommand{\StringTok}[1]{\textcolor[rgb]{0.31,0.60,0.02}{#1}}
\newcommand{\VariableTok}[1]{\textcolor[rgb]{0.00,0.00,0.00}{#1}}
\newcommand{\VerbatimStringTok}[1]{\textcolor[rgb]{0.31,0.60,0.02}{#1}}
\newcommand{\WarningTok}[1]{\textcolor[rgb]{0.56,0.35,0.01}{\textbf{\textit{#1}}}}
\usepackage{graphicx}
\makeatletter
\def\maxwidth{\ifdim\Gin@nat@width>\linewidth\linewidth\else\Gin@nat@width\fi}
\def\maxheight{\ifdim\Gin@nat@height>\textheight\textheight\else\Gin@nat@height\fi}
\makeatother
% Scale images if necessary, so that they will not overflow the page
% margins by default, and it is still possible to overwrite the defaults
% using explicit options in \includegraphics[width, height, ...]{}
\setkeys{Gin}{width=\maxwidth,height=\maxheight,keepaspectratio}
% Set default figure placement to htbp
\makeatletter
\def\fps@figure{htbp}
\makeatother
\setlength{\emergencystretch}{3em} % prevent overfull lines
\providecommand{\tightlist}{%
  \setlength{\itemsep}{0pt}\setlength{\parskip}{0pt}}
\setcounter{secnumdepth}{-\maxdimen} % remove section numbering
\ifLuaTeX
  \usepackage{selnolig}  % disable illegal ligatures
\fi
\usepackage{bookmark}
\IfFileExists{xurl.sty}{\usepackage{xurl}}{} % add URL line breaks if available
\urlstyle{same}
\hypersetup{
  pdftitle={NECN Weibull establishment parameters},
  pdfauthor={Sam Flake},
  hidelinks,
  pdfcreator={LaTeX via pandoc}}

\title{NECN Weibull establishment parameters}
\author{Sam Flake}
\date{2024-08-30}

\begin{document}
\maketitle

\section{LAI establishment curves}\label{lai-establishment-curves}

NECN v7 uses a modified Weibull curve to describe species shade
preferences, replacing the clunky LAI table that was found in the NECN
text input file in previous versions. The LAI table was based on expert
judgment, not data, and applied the same discontinuous LAI breaks to all
species. The new Weibull parameters are designed to be 1) more accurate
and customizable to each species, and 2) parameterized from data.

This script has an example for how to parameterize the Weibull
establishment curves for two species. I recommend, instead of running
this Rmd, look at the file \texttt{weibull\_params\_example.R}. For a
more detailed script (more similar to what you would need for a real
NECN model), see \texttt{get\_weibull\_params\_from\_fia\_all\_data.R}.
That script has more options to include additional species, deal with
rare species, and choose how the curves are fit. It should be pretty
much plug-and-play with new species or study areas with a little
modification (the species choices, matching them up to FIA SPCDs, and
your LAI parameters from the NECN functional table file).

\subsection{Establishment in NECN and a little Bayesian
tangent}\label{establishment-in-necn-and-a-little-bayesian-tangent}

During the reproduction phase of NECN, sites are evaluated for whether
or not a new cohort can be recruited there using several criteria. The
first hurdle which must be passed is whether there is enough light. For
each species, the LAI of the site determines the probability that a
recruit can establish there (this is what the LAI establishment table
and the Weibull parameters determine). So, what NECN needs, is a way to
represent the probability of recruitment given the LAI of the site, or
p(recruit\textbar LAI).

The way I've gone about doing this is by using inventory data from the
USFS Forest Inventory and Analysis (FIA) project. For each species, I
assume that the probability of recruitment at a given LAI value is
related to the probability of the FIA project finding a seedling in a
plot of that LAI -- it's a little different, but it should be strongly
and directly related. The easiest way to calculate something related to
establishment and LAI is to find where all the seedlings of a species
are located, and look at the LAI of those sites. However, instead of
giving us p(recruit\textbar LAI), this gives us p(LAI\textbar recruit)
-- what is the probability of a value of LAI, given that there are
seedlings there?

While p(LAI\textbar recruit) isn't quite right, we can relate it to
p(recruit\textbar LAI) by using Bayes' theorem:

p(recruit\textbar LAI) = p(LAI\textbar recruit) * p(recruit) / p(LAI)

While I don't use a Bayesian analysis here formally, I find the logic
compelling -- the probability of recruitment given LAI is related to the
histogram of LAIs where a seedling is found, by: 1) the landscape
distribution of LAI (what is the probability of a given LAI?), and 2)
the fecundity of the species (what is the probability of recruitment?).

Hopefully this motivating logic is apparent in the script below, when we
get to the steps that scale the probability of recruitment to set the
max p(recruit) equal for each species (which allows the area under the
curve, or fecundity, to differ among species) and when we limit the
landscape from which we sample (accounting for the distribution of LAI).

\subsection{Weibull overview}\label{weibull-overview}

The Weibull distribution is a probability distribution function
described by two parameters, a shape parameters and a scale parameter.
It is great for distributions that are always positive and have long
right tails -- like the typical distribution of leaf area index on a
landscape. By changing the shape and scale parameters, we can get a wide
variety of distributions, from a really shade-intolerant species with
shape parameter \textless{} 1, to a shade-tolerant with a shape
parameter around 2.

It has the following equation:

y = (a/b) * ((X/b)\^{}(a-1)) * exp(- (X/b)\^{}a)

where a is the shape parameter and b is the scale parameter. In our
case, X is plot-level LAI.

I've also modified the equation to allow for what I'm calling location
and adjustment parameters, c and d.~If you use these, then the curve is
not describing a PDF anymore, just a Weibull-ish curve that might fit
your data better. The area under the curve if you're using c or d will
probably not be 1, so some species might have greate reproductive
success than others -- this may be good or bad, depending on your needs.

y = (((a/b) * ((lai/b)\^{}(a-1)) * exp(- (lai/b)\^{}a)) + c) * d

Here's a set of 2-parameter Weibull distributions with varying shape
parameters and scale = 1.

\includegraphics{weibull_vignette_files/figure-latex/unnamed-chunk-1-1.pdf}

And here's the same with shape = 1 and scale varying.

\includegraphics{weibull_vignette_files/figure-latex/unnamed-chunk-2-1.pdf}

Here's a Weibull curve with shape = 2 and scale = 1.5, with c varying:

\includegraphics{weibull_vignette_files/figure-latex/unnamed-chunk-3-1.pdf}

And here's what the d parameter does, again with shape = 2 and scale =
1.5:

\includegraphics{weibull_vignette_files/figure-latex/unnamed-chunk-4-1.pdf}

So there's a variety of shapes we can fit using Weibull, which solves
one issue with the LAI tables, but we still need to be able to
parameterize them from real data.

\section{Paramaterizing light preferences for
species}\label{paramaterizing-light-preferences-for-species}

This script has several steps:

\begin{enumerate}
\def\labelenumi{\arabic{enumi}.}
\tightlist
\item
  import and wrangle FIA data for plots, trees, seedlings, and sitetrees
\item
  assemble FIA plots into age/species cohorts like LANDIS-II
\item
  estimate leaf area for plots using the same algorithm as NECN
\item
  get amount of regeneration per plot per species
\item
  fit distributions to regen\textasciitilde LAI values using nonlinear
  least squares
\end{enumerate}

\subsection{Step 0: Loading libraries and
functions}\label{step-0-loading-libraries-and-functions}

We also have a big functiont that fits the weibull parameters,
\texttt{fit\_weibull()}, but I won't put it here -- take a look at
get\_weibull\_params\_from\_fia\_all\_data.R for details.

We also have a few options that will make sense once we get through the
script. Mostly, we need to define what makes a plot ``available'' for
recruitment, so we're not including a bunch of plots out of the range of
a species where we wouldn't expect to find them. Don't worry about these
for now, but they're in the script for when you need them later.

\begin{Shaded}
\begin{Highlighting}[]
\CommentTok{\#should we restrict our analysis to plots that are near seedlings, near adult trees, or unrestricted?}
\CommentTok{\#this is by far the slowest part of the script}
\NormalTok{range\_method }\OtherTok{\textless{}{-}} \StringTok{"seedling"} \CommentTok{\#"seedling" or "adult"; other options skip this step and use all the data from the chosen states}
\CommentTok{\#range buffer size {-}{-} how many meters can plots be from seedlings to count in the calculation?}
\NormalTok{range\_buffer }\OtherTok{\textless{}{-}} \DecValTok{10000}
\CommentTok{\#should maximum suitability be set to 1 and everything scaled to match? This will (almost?)}
\CommentTok{\# always increase total suitable light levels and thus regeneration}
\NormalTok{scale\_max\_p }\OtherTok{\textless{}{-}} \ConstantTok{TRUE}
\CommentTok{\#should the area under the curve be set to 1 for all species? This will cancel out}
\CommentTok{\# differences in abundance/fecundity among species. Probably not recommended if you hae}
\CommentTok{\# a lot of range{-}restricted species, unless you\textquotesingle{}re using the range\_method option above.}
\CommentTok{\# But this is very much recommended if you\textquotesingle{}re wanting to fit a true Weibull PDF where AUC = 1 }
\CommentTok{\# (i.e. using use\_c\_d == "neither")}
\NormalTok{set\_auc\_to\_1 }\OtherTok{\textless{}{-}} \ConstantTok{FALSE}
\CommentTok{\#what formula should we use? See above equation fit\_weibull. For a regular Weibull PDF,}
\CommentTok{\# set use\_c\_d = "neither", to only use the shape and scale parameters. Otherwise,}
\CommentTok{\# you can use the c parameter to translate the whole curve up or the d parameter to stretch the}
\CommentTok{\# whole curve. This gives more flexibility in the kinds of shapes that can be fit,}
\CommentTok{\# and allows for the area under the curve to be greater than 1.}
\NormalTok{use\_c\_d }\OtherTok{\textless{}{-}} \StringTok{"d"}
\end{Highlighting}
\end{Shaded}

\section{Step 1: Wrangling data}\label{step-1-wrangling-data}

I've got some FIA data pre-wrangled for us. It's FIA plots randomly
selected from Michigan and Minnesota, and we're looking at establishment
of balsam fir (Abies balsamea) and quaking aspen (Populus tremuloides).
We need tree data to estimate LAI and seedling data to look at
recruitment. We also need the sitetree table to make some estimates of
tree ages.

\begin{Shaded}
\begin{Highlighting}[]
\CommentTok{\#species reference data}
\NormalTok{sp\_ref }\OtherTok{\textless{}{-}} \FunctionTok{read.csv}\NormalTok{(}\StringTok{"./example\_data/REF\_SPECIES.csv"}\NormalTok{)}

\NormalTok{trees }\OtherTok{\textless{}{-}} \FunctionTok{read.csv}\NormalTok{(}\StringTok{"./example\_data/trees.csv"}\NormalTok{)}
\NormalTok{plot }\OtherTok{\textless{}{-}} \FunctionTok{read.csv}\NormalTok{(}\StringTok{"./example\_data/plot.csv"}\NormalTok{)}
\NormalTok{seedlings }\OtherTok{\textless{}{-}} \FunctionTok{read.csv}\NormalTok{(}\StringTok{"./example\_data/seedlings.csv"}\NormalTok{)}
\NormalTok{sitetrees }\OtherTok{\textless{}{-}} \FunctionTok{read.csv}\NormalTok{(}\StringTok{"./example\_data/sitetrees.csv"}\NormalTok{)}
\end{Highlighting}
\end{Shaded}

Here's what our plots look like:

\includegraphics{weibull_vignette_files/figure-latex/unnamed-chunk-9-1.pdf}

\section{Step 2: Crosswalk LANDIS-II and
FIA}\label{step-2-crosswalk-landis-ii-and-fia}

this is the functional table and species table from your NECN project.
We need them for a few things, like calculating LAI and crosswalking
LANDIS-II species names and FIA SPCDs

\begin{Shaded}
\begin{Highlighting}[]
\NormalTok{func\_table }\OtherTok{\textless{}{-}} \FunctionTok{read.csv}\NormalTok{(}\StringTok{"./example\_data/NECN\_Functional\_Table\_inv\_moisture.csv"}\NormalTok{)}
\NormalTok{sp\_table }\OtherTok{\textless{}{-}} \FunctionTok{read.csv}\NormalTok{(}\StringTok{"./example\_data/NECN\_Spp\_Table\_inv\_necn7.csv"}\NormalTok{) }\SpecialCharTok{\%\textgreater{}\%}
  \FunctionTok{left\_join}\NormalTok{(func\_table, }\AttributeTok{by =} \StringTok{"FunctionalGroupIndex"}\NormalTok{) }
\end{Highlighting}
\end{Shaded}

We need to assign each FIA species group a value for KLAI and
MaximumLAI. You may have a lot of species in FIA that aren't in your
LANDIS project; I'd suggest just coming up with a catchall functional
group for those and putting all the extra SPGRPCDs in there. Multiple
SPGRPCDs might match each NECN functional group, and some FGs might not
have a corresponding SPGRPCD if there are no FIA species that match.
That's fine.

\begin{Shaded}
\begin{Highlighting}[]
\NormalTok{spgrp\_lai }\OtherTok{\textless{}{-}} \FunctionTok{data.frame}\NormalTok{(}\AttributeTok{SPGRPCD =} \FunctionTok{unique}\NormalTok{(trees}\SpecialCharTok{$}\NormalTok{SPGRPCD),}
                        \AttributeTok{FunctionalGroupIndex =} \FunctionTok{numeric}\NormalTok{(}\FunctionTok{length}\NormalTok{(}\FunctionTok{unique}\NormalTok{(trees}\SpecialCharTok{$}\NormalTok{SPGRPCD)))) }\SpecialCharTok{\%\textgreater{}\%}
  \FunctionTok{arrange}\NormalTok{(SPGRPCD) }\SpecialCharTok{\%\textgreater{}\%}
  \FunctionTok{mutate}\NormalTok{(}\AttributeTok{FunctionalGroupIndex =} \FunctionTok{ifelse}\NormalTok{(SPGRPCD }\SpecialCharTok{\%in\%} \FunctionTok{c}\NormalTok{(}\DecValTok{5}\NormalTok{,}\DecValTok{4}\NormalTok{, }\DecValTok{3}\NormalTok{, }\DecValTok{2}\NormalTok{), }\DecValTok{1}\NormalTok{, FunctionalGroupIndex))}\SpecialCharTok{\%\textgreater{}\%} \CommentTok{\#jack and red pine}
  \FunctionTok{mutate}\NormalTok{(}\AttributeTok{FunctionalGroupIndex =} \FunctionTok{ifelse}\NormalTok{(SPGRPCD }\SpecialCharTok{\%in\%} \FunctionTok{c}\NormalTok{(}\DecValTok{6}\NormalTok{), }\DecValTok{2}\NormalTok{, FunctionalGroupIndex))}\SpecialCharTok{\%\textgreater{}\%}   \CommentTok{\#balsam fir and white spruce}
  \FunctionTok{mutate}\NormalTok{(}\AttributeTok{FunctionalGroupIndex =} \FunctionTok{ifelse}\NormalTok{(SPGRPCD }\SpecialCharTok{\%in\%} \FunctionTok{c}\NormalTok{(}\DecValTok{9}\NormalTok{), }\DecValTok{3}\NormalTok{, FunctionalGroupIndex))}\SpecialCharTok{\%\textgreater{}\%}    \CommentTok{\#eastern white pine}
  \FunctionTok{mutate}\NormalTok{(}\AttributeTok{FunctionalGroupIndex =} \FunctionTok{ifelse}\NormalTok{(SPGRPCD }\SpecialCharTok{\%in\%} \FunctionTok{c}\NormalTok{(}\DecValTok{36}\NormalTok{, }\DecValTok{35}\NormalTok{), }\DecValTok{4}\NormalTok{, FunctionalGroupIndex)) }\SpecialCharTok{\%\textgreater{}\%} \CommentTok{\#black spruce}
  \FunctionTok{mutate}\NormalTok{(}\AttributeTok{FunctionalGroupIndex =} \FunctionTok{ifelse}\NormalTok{(SPGRPCD }\SpecialCharTok{\%in\%} \FunctionTok{c}\NormalTok{(}\DecValTok{7}\NormalTok{), }\DecValTok{5}\NormalTok{, FunctionalGroupIndex)) }\SpecialCharTok{\%\textgreater{}\%} \CommentTok{\#tamarack, ceder}
  \FunctionTok{mutate}\NormalTok{(}\AttributeTok{FunctionalGroupIndex =} \FunctionTok{ifelse}\NormalTok{(SPGRPCD }\SpecialCharTok{\%in\%} \FunctionTok{c}\NormalTok{(}\DecValTok{30}\NormalTok{, }\DecValTok{31}\NormalTok{, }\DecValTok{33}\NormalTok{, }\DecValTok{42}\NormalTok{), }\DecValTok{6}\NormalTok{, FunctionalGroupIndex)) }\SpecialCharTok{\%\textgreater{}\%} \CommentTok{\#northern hardwoods}
  \FunctionTok{mutate}\NormalTok{(}\AttributeTok{FunctionalGroupIndex =} \FunctionTok{ifelse}\NormalTok{(SPGRPCD }\SpecialCharTok{\%in\%} \FunctionTok{c}\NormalTok{(}\DecValTok{37}\NormalTok{, }\DecValTok{38}\NormalTok{, }\DecValTok{41}\NormalTok{), }\DecValTok{7}\NormalTok{, FunctionalGroupIndex)) }\SpecialCharTok{\%\textgreater{}\%} \CommentTok{\#aspen}
  \FunctionTok{mutate}\NormalTok{(}\AttributeTok{FunctionalGroupIndex =} \FunctionTok{ifelse}\NormalTok{(SPGRPCD }\SpecialCharTok{\%in\%} \FunctionTok{c}\NormalTok{(}\DecValTok{25}\NormalTok{, }\DecValTok{26}\NormalTok{, }\DecValTok{28}\NormalTok{, }\DecValTok{29}\NormalTok{, }\DecValTok{27}\NormalTok{, }\DecValTok{32}\NormalTok{, }\DecValTok{43}\NormalTok{, }\DecValTok{40}\NormalTok{, }\DecValTok{39}\NormalTok{, }\DecValTok{55}\NormalTok{), }\DecValTok{8}\NormalTok{, FunctionalGroupIndex)) }\SpecialCharTok{\%\textgreater{}\%} \CommentTok{\#southern harwoods}
  \CommentTok{\# mutate(FunctionalGroupIndex = ifelse(SPGRPCD \%in\% c(), 9, FunctionalGroupIndex)) \%\textgreater{}\% \#wet broadleaf }
  \CommentTok{\# mutate(FunctionalGroupIndex = ifelse(SPGRPCD \%in\% c(), 10, FunctionalGroupIndex)) \%\textgreater{}\% \#shrubs}
  \FunctionTok{left\_join}\NormalTok{(func\_table }\SpecialCharTok{\%\textgreater{}\%}\NormalTok{ dplyr}\SpecialCharTok{::}\FunctionTok{select}\NormalTok{(FunctionalGroupIndex, KLAI, MaximumLAI))}
\end{Highlighting}
\end{Shaded}

\begin{verbatim}
## Joining with `by = join_by(FunctionalGroupIndex)`
\end{verbatim}

\begin{Shaded}
\begin{Highlighting}[]
\CommentTok{\#here, we\textquotesingle{}re just doing ABBA and POTR, not all the species}
\NormalTok{spp\_to\_use\_all }\OtherTok{\textless{}{-}} \FunctionTok{c}\NormalTok{(}\StringTok{"ABBA"}\NormalTok{, }\StringTok{"POTR5"}\NormalTok{) }\CommentTok{\#sp\_table$SpeciesCode}
\end{Highlighting}
\end{Shaded}

We need to get SPCD for each species. This will be different depending
on how the species are named, but we want a crosswalk from the names
used in LANDIS to FIA SPCD somehow In this project, I used the USDA
PLANTS symbol, which is found in the FIA species reference table to
crosswalk to FIA species code. We want a column of SpeciesCode that is
your LANDIS species identifier, and a column of SPCD that is the FIA
species number.

\begin{Shaded}
\begin{Highlighting}[]
\NormalTok{sp\_ref}\SpecialCharTok{$}\NormalTok{SpeciesCode }\OtherTok{\textless{}{-}}\NormalTok{ sp\_ref}\SpecialCharTok{$}\NormalTok{SPECIES\_SYMBOL}
\NormalTok{spp\_crosswalk }\OtherTok{\textless{}{-}}\NormalTok{ sp\_ref[sp\_ref}\SpecialCharTok{$}\NormalTok{SPECIES\_SYMBOL }\SpecialCharTok{\%in\%}\NormalTok{ spp\_to\_use\_all, ] }\SpecialCharTok{\%\textgreater{}\%}
\NormalTok{  dplyr}\SpecialCharTok{::}\FunctionTok{arrange}\NormalTok{(SPECIES\_SYMBOL) }\SpecialCharTok{\%\textgreater{}\%}
\NormalTok{  dplyr}\SpecialCharTok{::}\FunctionTok{select}\NormalTok{(SpeciesCode, SPCD)}


\CommentTok{\# for seedlings, some species don\textquotesingle{}t have enough to get good estimates. This table}
\CommentTok{\# combines several SPCD for each species (e.g., uncommon Populus spp.) to }
\CommentTok{\# improve parameter estimates for rare species. You have to do this by hand for each}
\CommentTok{\# species you\textquotesingle{}re modeling. Here it\textquotesingle{}s simple because I\textquotesingle{}m just using two common species}
\NormalTok{spp\_crosswalk\_combine }\OtherTok{\textless{}{-}} \FunctionTok{tibble}\NormalTok{(}\AttributeTok{SpeciesCode =}\NormalTok{ spp\_to\_use\_all,}
                                \AttributeTok{SPCD =} \FunctionTok{list}\NormalTok{(}\DecValTok{12}\NormalTok{, }\CommentTok{\#Balsam fir}
                                            \DecValTok{746} \CommentTok{\#quaking aspen}
\NormalTok{                                ))}

\NormalTok{spp\_to\_use }\OtherTok{\textless{}{-}}\NormalTok{ spp\_crosswalk}\SpecialCharTok{$}\NormalTok{SpeciesCode}
\NormalTok{spcd\_to\_use }\OtherTok{\textless{}{-}}\NormalTok{ spp\_crosswalk}\SpecialCharTok{$}\NormalTok{SPCD}
\NormalTok{all\_spcd }\OtherTok{\textless{}{-}} \FunctionTok{unique}\NormalTok{(trees}\SpecialCharTok{$}\NormalTok{SPCD)}
\end{Highlighting}
\end{Shaded}

\section{Step 3: Calculate per-tree LAI contribution and plot-level
LAI}\label{step-3-calculate-per-tree-lai-contribution-and-plot-level-lai}

I'm using the same logic that NECN does to estimate plot-level LAI, so
that we don't have mismatches when we apply the parameters we're
generating to an NECN model. So we need to get ages and biomass for
cohorts, just like LANDIS-II.

\begin{Shaded}
\begin{Highlighting}[]
\CommentTok{\#we need to make sure the trees have ages so we can bin them}
\NormalTok{enough\_trees }\OtherTok{\textless{}{-}} \FunctionTok{table}\NormalTok{(sitetrees}\SpecialCharTok{$}\NormalTok{SPCD) }\SpecialCharTok{\%\textgreater{}\%} \StringTok{\textasciigrave{}}\AttributeTok{[}\StringTok{\textasciigrave{}}\NormalTok{(}\FunctionTok{which}\NormalTok{(}\FunctionTok{table}\NormalTok{(sitetrees}\SpecialCharTok{$}\NormalTok{SPCD) }\SpecialCharTok{\textgreater{}} \DecValTok{100}\NormalTok{)) }\SpecialCharTok{\%\textgreater{}\%} \FunctionTok{names}\NormalTok{()}

\CommentTok{\#make an age model for each species {-}{-} you could do this however you want; this is a quick and dirty}
\CommentTok{\# cubic regression for each species, and lumping all rare species into one model}
\NormalTok{age\_model }\OtherTok{\textless{}{-}} \FunctionTok{lm}\NormalTok{(}\FunctionTok{log}\NormalTok{(AGEDIA) }\SpecialCharTok{\textasciitilde{}} \FunctionTok{poly}\NormalTok{(}\FunctionTok{log}\NormalTok{(DIA), }\DecValTok{3}\NormalTok{)}\SpecialCharTok{*}\FunctionTok{as.factor}\NormalTok{(SPCD), }
                \AttributeTok{data =}\NormalTok{ sitetrees[}\SpecialCharTok{!}\FunctionTok{is.na}\NormalTok{(sitetrees}\SpecialCharTok{$}\NormalTok{DIA) }\SpecialCharTok{\&} \SpecialCharTok{!}\FunctionTok{is.na}\NormalTok{(sitetrees}\SpecialCharTok{$}\NormalTok{AGEDIA) }\SpecialCharTok{\&}\NormalTok{ sitetrees}\SpecialCharTok{$}\NormalTok{SPCD }\SpecialCharTok{\%in\%}\NormalTok{ enough\_trees, ])}
\NormalTok{age\_model2 }\OtherTok{\textless{}{-}} \FunctionTok{lm}\NormalTok{(}\FunctionTok{log}\NormalTok{(AGEDIA) }\SpecialCharTok{\textasciitilde{}} \FunctionTok{poly}\NormalTok{(}\FunctionTok{log}\NormalTok{(DIA), }\DecValTok{3}\NormalTok{), }
                 \AttributeTok{data =}\NormalTok{ sitetrees[}\SpecialCharTok{!}\FunctionTok{is.na}\NormalTok{(sitetrees}\SpecialCharTok{$}\NormalTok{DIA) }\SpecialCharTok{\&} \SpecialCharTok{!}\FunctionTok{is.na}\NormalTok{(sitetrees}\SpecialCharTok{$}\NormalTok{AGEDIA), ])}

\CommentTok{\#asign ages to trees based on species and diameter}
\NormalTok{trees }\OtherTok{\textless{}{-}}\NormalTok{ trees }\SpecialCharTok{\%\textgreater{}\%}
  \FunctionTok{mutate}\NormalTok{(}\AttributeTok{PLOT.YEAR =} \FunctionTok{paste}\NormalTok{(PLT\_CN, INVYR, }\AttributeTok{sep=}\StringTok{"."}\NormalTok{)) }\SpecialCharTok{\%\textgreater{}\%}
  \FunctionTok{right\_join}\NormalTok{(., plot, }\AttributeTok{by =} \FunctionTok{c}\NormalTok{(}\StringTok{"PLT\_CN"} \OtherTok{=} \StringTok{"CN"}\NormalTok{)) }\SpecialCharTok{\%\textgreater{}\%}
\NormalTok{  dplyr}\SpecialCharTok{::}\FunctionTok{mutate}\NormalTok{(}\AttributeTok{DIA\_cm =}\NormalTok{ DIA }\SpecialCharTok{*} \FloatTok{2.54}\NormalTok{,}
                \AttributeTok{HT\_m =}\NormalTok{ HT }\SpecialCharTok{/} \FloatTok{3.3808}\NormalTok{) }\SpecialCharTok{\%\textgreater{}\%}
\NormalTok{  dplyr}\SpecialCharTok{::}\FunctionTok{filter}\NormalTok{(STATUSCD }\SpecialCharTok{==} \DecValTok{1}\NormalTok{) }\SpecialCharTok{\%\textgreater{}\%}
\NormalTok{  dplyr}\SpecialCharTok{::}\FunctionTok{left\_join}\NormalTok{(spgrp\_lai)}
\end{Highlighting}
\end{Shaded}

\begin{verbatim}
## Joining with `by = join_by(SPGRPCD)`
\end{verbatim}

\begin{Shaded}
\begin{Highlighting}[]
\ControlFlowTok{for}\NormalTok{(i }\ControlFlowTok{in} \DecValTok{1}\SpecialCharTok{:}\FunctionTok{length}\NormalTok{(all\_spcd))\{}
  
\NormalTok{  spcd }\OtherTok{=}\NormalTok{ all\_spcd[i]}
\NormalTok{  sp }\OtherTok{=}\NormalTok{ spp\_crosswalk[}\FunctionTok{match}\NormalTok{(spcd, spp\_crosswalk}\SpecialCharTok{$}\NormalTok{SPCD), }\StringTok{"SpeciesCode"}\NormalTok{]}
  
  \ControlFlowTok{if}\NormalTok{(spcd }\SpecialCharTok{\%in\%}\NormalTok{ enough\_trees)\{}
    
\NormalTok{    trees[trees}\SpecialCharTok{$}\NormalTok{SPCD }\SpecialCharTok{==}\NormalTok{ spcd, }\StringTok{"TOTAGE2"}\NormalTok{] }\OtherTok{\textless{}{-}} \FunctionTok{exp}\NormalTok{(}\FunctionTok{predict}\NormalTok{(age\_model,}
                                                        \AttributeTok{newdata =}\NormalTok{ trees[trees}\SpecialCharTok{$}\NormalTok{SPCD }\SpecialCharTok{==}\NormalTok{ spcd, ]))}
    
\NormalTok{  \} }\ControlFlowTok{else}\NormalTok{\{}
\NormalTok{    trees[trees}\SpecialCharTok{$}\NormalTok{SPCD }\SpecialCharTok{==}\NormalTok{ spcd, }\StringTok{"TOTAGE2"}\NormalTok{] }\OtherTok{\textless{}{-}} \FunctionTok{exp}\NormalTok{(}\FunctionTok{predict}\NormalTok{(age\_model2,}
                                                        \AttributeTok{newdata =}\NormalTok{ trees[trees}\SpecialCharTok{$}\NormalTok{SPCD }\SpecialCharTok{==}\NormalTok{ spcd, ]))}
    
\NormalTok{  \}}
\NormalTok{\}}

\NormalTok{trees}\SpecialCharTok{$}\NormalTok{age }\OtherTok{\textless{}{-}} \FunctionTok{ifelse}\NormalTok{(}\FunctionTok{is.na}\NormalTok{(trees}\SpecialCharTok{$}\NormalTok{TOTAGE), trees}\SpecialCharTok{$}\NormalTok{TOTAGE2, trees}\SpecialCharTok{$}\NormalTok{TOTAGE)}
\NormalTok{trees}\SpecialCharTok{$}\NormalTok{age }\OtherTok{\textless{}{-}} \FunctionTok{ifelse}\NormalTok{(trees}\SpecialCharTok{$}\NormalTok{age }\SpecialCharTok{\textgreater{}} \DecValTok{500}\NormalTok{, }\DecValTok{500}\NormalTok{, trees}\SpecialCharTok{$}\NormalTok{age)}
\NormalTok{breaks }\OtherTok{\textless{}{-}} \FunctionTok{seq}\NormalTok{(}\DecValTok{0}\NormalTok{, }\FunctionTok{max}\NormalTok{(trees}\SpecialCharTok{$}\NormalTok{age, }\AttributeTok{na.rm =} \ConstantTok{TRUE}\NormalTok{) }\SpecialCharTok{+}\NormalTok{ (}\DecValTok{10} \SpecialCharTok{{-}} \FunctionTok{max}\NormalTok{(trees}\SpecialCharTok{$}\NormalTok{age, }\AttributeTok{na.rm =} \ConstantTok{TRUE}\NormalTok{) }\SpecialCharTok{\%\%} \DecValTok{10}\NormalTok{), }\AttributeTok{by =} \DecValTok{5}\NormalTok{)}
\NormalTok{trees}\SpecialCharTok{$}\NormalTok{age\_bin }\OtherTok{\textless{}{-}}\NormalTok{ base}\SpecialCharTok{::}\FunctionTok{cut}\NormalTok{(trees}\SpecialCharTok{$}\NormalTok{age, }\AttributeTok{breaks =}\NormalTok{ breaks, }\AttributeTok{labels =}\NormalTok{ breaks[}\SpecialCharTok{{-}}\DecValTok{1}\NormalTok{], }\AttributeTok{right =} \ConstantTok{TRUE}\NormalTok{)}

\CommentTok{\#use the NECN LAI calculation to get plot{-}level LAI}
\NormalTok{trees\_bin }\OtherTok{\textless{}{-}}\NormalTok{ trees }\SpecialCharTok{\%\textgreater{}\%}
  \FunctionTok{group\_by}\NormalTok{(PLOT.YEAR, SPCD, age\_bin) }\SpecialCharTok{\%\textgreater{}\%}
  \FunctionTok{summarise}\NormalTok{(}\AttributeTok{cohort\_biomass =} \FunctionTok{sum}\NormalTok{(CARBON\_AG}\SpecialCharTok{/}\FloatTok{0.47}\NormalTok{, }\AttributeTok{na.rm =} \ConstantTok{TRUE}\NormalTok{) }\SpecialCharTok{*} \FloatTok{0.11}\NormalTok{, }\CommentTok{\#sum biomass and convert to g m{-}2}
            \AttributeTok{KLAI =}\NormalTok{ KLAI[}\DecValTok{1}\NormalTok{],}
            \AttributeTok{MaximumLAI =}\NormalTok{ MaximumLAI[}\DecValTok{1}\NormalTok{]) }\SpecialCharTok{\%\textgreater{}\%}
  \FunctionTok{mutate}\NormalTok{(}\AttributeTok{LAI\_tree =}\NormalTok{ MaximumLAI }\SpecialCharTok{*}\NormalTok{ cohort\_biomass}\SpecialCharTok{/}\NormalTok{(KLAI }\SpecialCharTok{+}\NormalTok{ cohort\_biomass))}
\end{Highlighting}
\end{Shaded}

\begin{verbatim}
## `summarise()` has grouped output by 'PLOT.YEAR', 'SPCD'. You can override using
## the `.groups` argument.
\end{verbatim}

\begin{Shaded}
\begin{Highlighting}[]
\CommentTok{\#calculate leaf area index per plot}
\NormalTok{plot\_leaf\_area }\OtherTok{\textless{}{-}}\NormalTok{ trees\_bin }\SpecialCharTok{\%\textgreater{}\%}
  \FunctionTok{group\_by}\NormalTok{(PLOT.YEAR) }\SpecialCharTok{\%\textgreater{}\%}
  \CommentTok{\#m2 per tree * trees per acre * acre per m2 = meters squared leaf area per meter squared ground}
  \FunctionTok{summarise}\NormalTok{(}\AttributeTok{LAI =} \FunctionTok{sum}\NormalTok{(LAI\_tree)) }\SpecialCharTok{\%\textgreater{}\%}
  \FunctionTok{filter}\NormalTok{(}\SpecialCharTok{!}\FunctionTok{is.na}\NormalTok{(LAI) }\SpecialCharTok{\&} \SpecialCharTok{!}\FunctionTok{is.infinite}\NormalTok{(LAI)) }\SpecialCharTok{\%\textgreater{}\%}
  \FunctionTok{filter}\NormalTok{(LAI }\SpecialCharTok{\textless{}} \DecValTok{20}\NormalTok{)}

\FunctionTok{hist}\NormalTok{(plot\_leaf\_area}\SpecialCharTok{$}\NormalTok{LAI)}
\end{Highlighting}
\end{Shaded}

\includegraphics{weibull_vignette_files/figure-latex/unnamed-chunk-13-1.pdf}

\section{Step 4: See where seedlings are present and which plots they
could be
present}\label{step-4-see-where-seedlings-are-present-and-which-plots-they-could-be-present}

We could be done at this point, if we wanted -- just join the seedlings
to the LAI table and make a histogram and call it good. I think it's
working better to go a little further, and spatially restrict the plots
we're considering as potential for recruitment.

First let's get how many seedlings we have per plot

\begin{Shaded}
\begin{Highlighting}[]
\CommentTok{\# get seedlings}
\NormalTok{seedlings }\OtherTok{\textless{}{-}}\NormalTok{ seedlings }\SpecialCharTok{\%\textgreater{}\%}
  \FunctionTok{mutate}\NormalTok{(}\AttributeTok{PLOT.YEAR =} \FunctionTok{paste}\NormalTok{(PLT\_CN, INVYR, }\AttributeTok{sep=}\StringTok{"."}\NormalTok{),}
         \AttributeTok{TPA\_UNADJ =} \FunctionTok{ifelse}\NormalTok{(}\FunctionTok{is.na}\NormalTok{(TPA\_UNADJ), }\DecValTok{0}\NormalTok{, TPA\_UNADJ)) }\SpecialCharTok{\%\textgreater{}\%}
  \FunctionTok{filter}\NormalTok{(PLOT.YEAR }\SpecialCharTok{\%in\%}\NormalTok{ plot\_leaf\_area}\SpecialCharTok{$}\NormalTok{PLOT.YEAR) }\SpecialCharTok{\%\textgreater{}\%} \CommentTok{\#filter out plots with bad LAI data, inappropriate COND, etc.}
  \CommentTok{\# filter(TOTAGE \textless{}= 5) \%\textgreater{}\% \#only useful in RMRS zone, and only collected fora  subset of trees}
  \FunctionTok{group\_by}\NormalTok{(PLOT.YEAR) }\SpecialCharTok{\%\textgreater{}\%}
  \FunctionTok{mutate}\NormalTok{(}\AttributeTok{SEEDLING\_COUNT =} \FunctionTok{sum}\NormalTok{(TPA\_UNADJ)) }\SpecialCharTok{\%\textgreater{}\%}
  \FunctionTok{slice\_head}\NormalTok{(}\AttributeTok{n =} \DecValTok{1}\NormalTok{)}

\FunctionTok{hist}\NormalTok{(seedlings}\SpecialCharTok{$}\NormalTok{SEEDLING\_COUNT)}
\end{Highlighting}
\end{Shaded}

\includegraphics{weibull_vignette_files/figure-latex/unnamed-chunk-14-1.pdf}

Next we need to figure out which plots are available for recruitment for
each species. Here, I'm restricting the available plots to those within
10 km of a seedling, as a way to represent the geographical range of the
species' natural recruitment. There's other options in the other script.
You could get very fancy with this part if you wanted to.

First let's do this with balsam fir to demonstrate:

\begin{Shaded}
\begin{Highlighting}[]
\NormalTok{  SPCD }\OtherTok{\textless{}{-}}\NormalTok{ spp\_crosswalk\_combine[[}\DecValTok{1}\NormalTok{, }\StringTok{"SPCD"}\NormalTok{]][[}\DecValTok{1}\NormalTok{]]}
  
\NormalTok{  seedling\_sub }\OtherTok{\textless{}{-}}\NormalTok{ seedlings[seedlings}\SpecialCharTok{$}\NormalTok{SPCD }\SpecialCharTok{\%in\%}\NormalTok{ SPCD,]}
  
\NormalTok{  plot\_sf }\OtherTok{\textless{}{-}}\NormalTok{ plot  }\SpecialCharTok{\%\textgreater{}\%}
\NormalTok{    sf}\SpecialCharTok{::}\FunctionTok{st\_as\_sf}\NormalTok{(}\AttributeTok{coords =} \FunctionTok{c}\NormalTok{(}\StringTok{"LON"}\NormalTok{, }\StringTok{"LAT"}\NormalTok{)) }\SpecialCharTok{\%\textgreater{}\%}
\NormalTok{    st\_as\_sf }\SpecialCharTok{\%\textgreater{}\%}
\NormalTok{    sf}\SpecialCharTok{::}\FunctionTok{st\_set\_crs}\NormalTok{(}\StringTok{"EPSG:4326"}\NormalTok{) }\SpecialCharTok{\%\textgreater{}\%}
\NormalTok{    sf}\SpecialCharTok{::}\FunctionTok{st\_transform}\NormalTok{(}\StringTok{"EPSG:5070"}\NormalTok{) }\SpecialCharTok{\%\textgreater{}\%}
    \FunctionTok{mutate}\NormalTok{(}\AttributeTok{PLOT.YEAR =} \FunctionTok{paste}\NormalTok{(CN, INVYR, }\AttributeTok{sep=}\StringTok{"."}\NormalTok{))}
    
\NormalTok{  seedling\_sf }\OtherTok{\textless{}{-}} \FunctionTok{left\_join}\NormalTok{(seedling\_sub, }\FunctionTok{select}\NormalTok{(plot, CN, LAT, LON), }\AttributeTok{by =} \FunctionTok{c}\NormalTok{(}\StringTok{"PLT\_CN"} \OtherTok{=} \StringTok{"CN"}\NormalTok{)) }\SpecialCharTok{\%\textgreater{}\%}
\NormalTok{    sf}\SpecialCharTok{::}\FunctionTok{st\_as\_sf}\NormalTok{(}\AttributeTok{coords =} \FunctionTok{c}\NormalTok{(}\StringTok{"LON"}\NormalTok{, }\StringTok{"LAT"}\NormalTok{))}\SpecialCharTok{\%\textgreater{}\%}
\NormalTok{    sf}\SpecialCharTok{::}\FunctionTok{st\_set\_crs}\NormalTok{(}\StringTok{"EPSG:4326"}\NormalTok{) }\SpecialCharTok{\%\textgreater{}\%}
\NormalTok{    sf}\SpecialCharTok{::}\FunctionTok{st\_transform}\NormalTok{(}\StringTok{"EPSG:5070"}\NormalTok{)}
  
  \CommentTok{\#create a zone within buffer distance of seedlings}
\NormalTok{  seedling\_buffer }\OtherTok{\textless{}{-}}\NormalTok{ sf}\SpecialCharTok{::}\FunctionTok{st\_buffer}\NormalTok{(seedling\_sf, range\_buffer) }\SpecialCharTok{\%\textgreater{}\%} 
    \FunctionTok{st\_union}\NormalTok{() }\SpecialCharTok{\%\textgreater{}\%} 
    \FunctionTok{st\_as\_sf}\NormalTok{()}
  
  \CommentTok{\#figure out which plots are in range}
\NormalTok{  plot\_sf}\SpecialCharTok{$}\NormalTok{in\_range }\OtherTok{\textless{}{-}} \FunctionTok{lengths}\NormalTok{(}\FunctionTok{st\_within}\NormalTok{(plot\_sf, seedling\_buffer))}
\end{Highlighting}
\end{Shaded}

Here's what all of our plots look like:

\begin{Shaded}
\begin{Highlighting}[]
\FunctionTok{plot}\NormalTok{(}\FunctionTok{st\_geometry}\NormalTok{(plot\_sf))}
\end{Highlighting}
\end{Shaded}

\includegraphics{weibull_vignette_files/figure-latex/unnamed-chunk-16-1.pdf}

But here's where balsam fir regeneration occurs -- just the Upper
Penninsula of Michigan and northern Minnesota, mostly. This is the 10 km
buffer we created around all the plots with balsam fir seedlings.

\begin{Shaded}
\begin{Highlighting}[]
\FunctionTok{plot}\NormalTok{(}\FunctionTok{st\_geometry}\NormalTok{(plot\_sf))}
\FunctionTok{plot}\NormalTok{(}\FunctionTok{st\_geometry}\NormalTok{(seedling\_buffer), }\AttributeTok{col =} \StringTok{"red"}\NormalTok{, }\AttributeTok{add =} \ConstantTok{TRUE}\NormalTok{)}
\end{Highlighting}
\end{Shaded}

\includegraphics{weibull_vignette_files/figure-latex/unnamed-chunk-17-1.pdf}

So we want to just use a subset of the plots:

\begin{Shaded}
\begin{Highlighting}[]
\FunctionTok{plot}\NormalTok{(plot\_sf[}\StringTok{"in\_range"}\NormalTok{])}
\end{Highlighting}
\end{Shaded}

\includegraphics{weibull_vignette_files/figure-latex/unnamed-chunk-18-1.pdf}
When we sum up the number of seedlings in each plot, we get something
like this, where we have a column of plot IDs, a column of LAI for each
plot, and then the number of seedlings for each species (in this case,
just balsam fir). We've counted seedlings when they're present, assigned
a 0 if the plot is in range and has no seedlings, and an NA if it's out
of range.

\begin{verbatim}
## # A tibble: 6 x 3
##   PLOT.YEAR                LAI  ABBA
##   <chr>                  <dbl> <dbl>
## 1 100180926010661.2005 0.00941     0
## 2 100181596010661.2005 1.54        0
## 3 100181813010661.2005 2.55        3
## 4 100182567010661.2005 0.786       0
## 5 100183466010661.2005 2.38       NA
## 6 100184322010661.2005 4.01       NA
\end{verbatim}

\section{Step 5: Fit a distribution to the seedling
data}\label{step-5-fit-a-distribution-to-the-seedling-data}

Now that we know the LAI for a bunch of plots and whether or not there's
seedlings present in those plots, we can start to explore how seedlings
are distributed with respect to LAI and try to fit distributions to that
data. Here, I've binned LAI into 30 equal bins, and, for each bin, found
what proportion of the plots have balsam fir seedlings:

\includegraphics{weibull_vignette_files/figure-latex/unnamed-chunk-21-1.pdf}

We can see that there's a peak in seedlings at around LAI = 4 or so,
which is a little past the point where most sorts of forests' canopies
have closed. This pattern of regeneration is what we expect from a
shade-tolerant tree like balsam fir. There are still some seedlings in
more open sites, perhaps advanced regeneration in a gap or just early
colonists of a blowdown or fire.

We might want to fiddle with this data some, depending on how our model
is parameterized. We could set the sum of the area under the curve equal
to 1, which would make this closer to a PDF, or we could scale the whole
thing so that the maximum probability of finding a seedling is equal to
1, which more closely matches the behavior of the LAI-establishment
tables from NECN v6.

\includegraphics{weibull_vignette_files/figure-latex/unnamed-chunk-22-1.pdf}

Note that things get a little funky at higher LAIs -- that's because
there aren't that many plots there. We can account for that when we fit
our distributions by including the number of plots in each bin as
weights, which will downweight the highly variable right tail of that
distribution and put more emphasis on fitting the left of the
distribution where most of the data are. Here's how those weights look:

\includegraphics{weibull_vignette_files/figure-latex/unnamed-chunk-23-1.pdf}

Now that we have the probability of finding a seedling at a given LAI,
we can fit a curve to that data and generate the inputs we need for
NECN. I've found that using that a, b, and d parameters works pretty
well for this species, and I'll be using the version of the seedling
data that scales the maximum p(seedling) to equal 1, so I can compare to
the NECN v6 values more easily.

\begin{verbatim}
## [[1]]
\end{verbatim}

\includegraphics{weibull_vignette_files/figure-latex/unnamed-chunk-24-1.pdf}

Let's compare this to a previous version of the model, based on
parameters inherited from a previous model. Those parameters looked like
this:

\begin{verbatim}
>> Shade    Max LAI
>> Class    
>> ------------------------------   
    1     1 
    2     2.5 
    3     3.5 
    4     6
    5     8

LightEstablishmentTable 
                    
>>  Spp Shade   Probability                 
>>  Class       by Actual Shade                 
>>   -----------------------------                  
>>      0       1     2     3       4       5
    1   1.0   0.5   0.02    0.0   0.0     0.0
    2     0.6     1.0   0.5   0.02  0.0   0.0
    3     0.01  0.6   1.0     0.5     0.02  0.0
    4     0.0     0.01  0.6   1.0     0.5     0.02
    5     0.0     0.0     0.01  0.5   1.0     1.0
\end{verbatim}

We had balsam fir as shade tolerance class 4, which means its
recruitment would have peaked at LAI between 3.5 and 6. We can look at
the FIA data and see that this is a little off. Let's plot the original
LAI table values as blue horizontal lines, on top of the curve we just
calibrated, and see how it compares.

\includegraphics{weibull_vignette_files/figure-latex/unnamed-chunk-25-1.pdf}

We're actually not that far off in this case -- the shape of the curve
is similar, and the LAI of maximum recruitment success isn't that
dissimilar. But we've now got a repeatable workflow, based on real data,
that we can reference. We can quickly generate curves for new species
rather than copying from projects from dissimilar areas.

As far as parameters go, we end up with a shape parameter of
\texttt{\{r\}mod\_output{[}{[}1{]}{]}\$shape}, a scale parameter of
\texttt{\{r\}mod\_output{[}{[}1{]}{]}\$scale}, and an adjust parameter
of \texttt{\{r\}mod\_output{[}{[}1{]}{]}\$adjust}. Because I set the
script not to use the location parameter, it is set to 0 (so the right
tail asymptotically approaches 0). The shape parameter greater than 1
means that the species has a peak in recruitment at LAI \textgreater{}
0; a really shade-intolerant species should have a negative-exponential
shape, with shape parameter less than 1. Let's take a look at one and
make sure that this all works like we expect.

\subsubsection{Adding a species}\label{adding-a-species}

Let's do the same thing for a shade-intolerant species, quaking aspen.
I'll just run through the whole script with both species, balsam fir
(ABBA) and quaking aspen so we can compare.

\includegraphics{weibull_vignette_files/figure-latex/unnamed-chunk-26-1.pdf}

\begin{verbatim}
## [[1]]
\end{verbatim}

\includegraphics{weibull_vignette_files/figure-latex/unnamed-chunk-26-2.pdf}

\begin{verbatim}
## 
## [[2]]
\end{verbatim}

\includegraphics{weibull_vignette_files/figure-latex/unnamed-chunk-26-3.pdf}

Looking at the previous two panels, we can see that the distribution of
recruits in FIA plots differ strongly between balsam fir and quaking
aspen, and that our Weibull curves were able to represent them pretty
well. As far as the parameters go, they look like this:

\begin{Shaded}
\begin{Highlighting}[]
\FunctionTok{print}\NormalTok{(dplyr}\SpecialCharTok{::}\FunctionTok{select}\NormalTok{(weibull\_models, Species, shape, scale, location, adjust))}
\end{Highlighting}
\end{Shaded}

\begin{verbatim}
## # A tibble: 2 x 5
## # Rowwise:  Species
##   Species shape scale location adjust
##   <chr>   <dbl> <dbl>    <dbl>  <dbl>
## 1 ABBA     1.64  5.60        0   7.00
## 2 POTR5    1     7.88        0   7.73
\end{verbatim}

The shape parameter is the most important to determining whether the
species has its highest or lowest recruitment at LAI = 0. You may need
to manually set that for some species to get the right shape. Here, the
curve-fitting algorithm got it right, but if it doesn't, you can modify
the species\_bounds variable in the script.

This script can be used for any number of species, as long as they're in
FIA. It would take a little work to adapt it to other plot networks, but
it shouldn't be that difficult -- the main steps would remain the same.
If no plot data is available, of course, using inherited parameters from
other sites or similar species remains one of the easiest ways to get a
model up and running so it can be calibrated. Let me know if you need
any help with this script or NECN regeneration in general, at
\href{mailto:swflake@ncsu.edu}{\nolinkurl{swflake@ncsu.edu}}.

\end{document}
